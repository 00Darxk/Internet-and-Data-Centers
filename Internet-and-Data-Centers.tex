\documentclass{article}

\usepackage{cancel}
\usepackage{amsmath,amssymb}
\usepackage[includehead,nomarginpar]{geometry}
\usepackage{graphicx}
\usepackage{amsfonts} 
\usepackage{verbatim}
\usepackage{mathrsfs}  
\usepackage{lmodern}
\usepackage{braket}
\usepackage{bookmark}
\usepackage{fancyhdr}
\usepackage{romanbarpagenumber}
\usepackage{minted}
%\usepackage{subfig}
\usepackage[italian]{babel}
\usepackage{float}
%\usepackage{wrapfig}
%\usepackage[export]{adjustbox}
\usepackage{contour}
\usepackage[normalem]{ulem}
\usepackage{svg} % to include svg images
\allowdisplaybreaks

\setlength{\headheight}{12.0pt}
\addtolength{\topmargin}{-12.0pt}
\graphicspath{ {./Immagini/} }

\hypersetup{
    colorlinks=true,
    linkcolor=black,
    pdftitle={Appunti di Internet and Data Centers},
    pdfauthor={Giacomo Sturm},
    pdfsubject={Internet and Data Centers},
    pdfkeywords={}
}

\newsavebox{\tempbox} %{\raisebox{\dimexpr.5\ht\tempbox-.5\height\relax}}


\makeatother
\renewcommand{\contentsname}{Indice}
\numberwithin{equation}{subsection}
\newcommand{\tageq}{\tag{\stepcounter{equation}\theequation}}
\AtBeginDocument{%
    \renewcommand{\figurename}{Fig.}
}
\renewcommand{\ULdepth}{1.8pt}
\contourlength{0.6pt}
\newcommand{\myuline}[1]{%
    \uline{\phantom{#1}}%
    \llap{\contour{white}{#1}}%
}
\fancypagestyle{link}{\fancyhf{}\renewcommand{\headrulewidth}{0pt}\fancyfoot[C]{Sorgente del file \LaTeX\space ed ultima versione del testo disponibile al link: \url{https://github.com/00Darxk/Internet-and-Data-Centers/}}}

\begin{document}

\title{%
    \textbf{Internet and Data Centers}  \\ 
    \large Appunti delle Lezioni di Internet and Data Centers \\
    \textit{Anno Accademico: 2025/26}}
\author{\textit{Giacomo Sturm}}
\date{\textit{Dipartimento di Ingegneria Civile, Informatica e delle Tecnologie Aeronautiche \\
Università degli Studi ``Roma Tre"}}

\maketitle
\thispagestyle{link}

\clearpage


\pagestyle{fancy}
\fancyhead{}\fancyfoot{}
\fancyhead[C]{\textit{Internet and Data Centers - Università degli Studi ``Roma Tre"}}
\fancyfoot[C]{\thepage}
\pagenumbering{Roman}

\tableofcontents

\clearpage
\pagenumbering{arabic}

\section{Introduzione: ISPs ed IXPs}

%% yapping 

Gli \textit{Internet Service Provider} ISP o \textit{Autonomous System} sono gestori autonomi della rete che hanno una responsabilità su un certo territorio. Su queste reti indipendenti viaggiano i pacchetti, gli ISP sono infatti responsabili di inoltrare i pacchetti che passano al loro interno per raggiungere la destinazione. 
In totale sono presenti circa 170000 ISP al mondo, alcuni di questi sono pubblicizzate e pubbliche per offrire servizi ad utenti comuni, altri offrono servizi a grandi compagni o altri ISP. 
Questi ISP formano una gerarchia, non dichiarata, che si può inferire, non essendo questi ISP governati da una struttura o organizzazione sovrastante. Alcuni di questi ISP sono privati e non si mostrano, per cui è possibile solamente inferire la gerarchia considerando questi ISP privati. Ogni ISP si cura solamente dei suoi interessi specifici, attraverso i loro clienti, permettono di connettersi ad un prezzo economico con altri servizi o utenti nel loro territorio. Questi ISP pagano altri ISP più grandi per raggiungere obiettivi sempre più remoti, in questo modo si può determinare una gerarchia che parte dagli ISP più piccoli ai più grandi. 

Queste regioni geografiche che identificano i territori di un certo ISP non sono separate, ma sono fortemente sovrapposte, è solamente una divisione logica e parzialmente può essere interpretata come una suddivisione geografica. 

Questi ISP comunicano tra di loro attraverso una connessione privata di rete \textit{Private Network Interconnects} PNI, queste possono essere dei cavi fisici di proprietà degli ISP oppure possono essere affittati, che collegano due router tra questi due ISP. Il costo per questa PNI viene pagato in base ad accordi privati tra i vari ISP ed in base al traffico. Questo è un rapporto commerciale a varie livelli, ci può essere un cliente che paga per utilizzare i servizi di un altro ISP, oppure bilaterale o peer-to-peer, dove i due ISP si scambiano pacchetti destinati all'altro ISP, ed pagano entrambi insieme il costo del PNI. 

Generalmente si paga sul picco di traffico giornaliero. 


Questo processo tuttavia non è più efficiente, con il concetto di economia di scala, invece di stendere fili singoli tra tutti gli ISP, si creano di punti di scambio comuni dove tutti i provider portano una loro macchina ed un filo che la collega alla loro rete, a questo punto per comunicare con altri ISP si crea una connessione logica con uno degli ISP presenti nel punto di scambio. Questo si chiama \textit{Internet eXchange Points} IXP. Questi sono governati da associazioni che mettono a servizio dei propri consorziati questi punti di scambio in modo sicuro. 
Quando si offre un servizio si cominciano ad offrirne di ulteriori, quindi oltre alla comunicazione offrono servizi di housing, di computing, etc. formando i tradizionali \textit{Data Centers}. 
In questi punti di scambio ci sono interconnessioni di diverso tipo, commerciali, peer-to-peer, gratuiti, ma sempre offerti nello stesso punto di scambio. 
Al mondo sono presenti circa un migliaio di questi IXP, in Europa il più grande è ad Amsterdam: AMS-IX. 

La rete di reti non è più una rete di macchine connesse tra di loro, ma coagula considerando questi punti di scambio comuni dove sono presenti delle macchine. Un \textit{Data Center} è uno spazio dedicato contenente computer, risorse di massa e sistemi di telecomunicazione. Questi si trovano generalmente in scantinati, ed i più grandi in zone dove l'energia costa poco o è facilmente procurable, oppure in zone dove la temperatura è pressoché bassa, per diminuire il costo di raffreddamento. In questi data center si brucia una quantità molto elevata di energia. 
Per l'importanza di questi data center, sono costantemente sorvegliati e protetti. Ogni macchina ha una procedura di recupero e sono pubbliche e quindi si assume che chi è in grado di avvicinarsi ad un macchina può effettuare queste procedure di recovery ed entrare in possesso della macchina. Quindi la sicurezza deve essere anche fisica per impedire alle persone di avvicinarsi a questi macchinari. 

C'è una differenza tra l'\textit{hosting} e l'\textit{housing}, nel primo l'hardware è posseduto dal data center, mentre nel secondo l'hardware non è fornito dal data center me viene procurato dal cliente. 
Altri centri invece hanno un solo cliente che è il proprietario, questi generalmente vengono realizzati solo dai più grandi come Google, Microsoft, etc. 


Questa tendenza è nata negli ultimi decenni, quando oltre ad offrire connettività gli ISP cominciarono ad offrire hosting ed housing, creando propri data center. La connettività è ormai diventata un bene di consumo a basso costo, quasi dato per scontato. 
Questi data center possono essere interni ad una singola rete o condivisi tra varie reti. 

Dentro un IXP si può facilmente inserire un data center, avendo già la struttura e l'organizzazione per ospitarli, unendo i router e macchine dei vari ISP a risorse di calcolo offerte dal data center. 

%% !! Cloud

La definizione di \textit{cloud} fornita dal NIST, \textit{National Institute of Standard and Technology}, è caratterizzata dalla possibilità di scalare su richiesta, un accesso a banda capiente, la possibilità di attivare risorse su richiesta, risposta rapida e misurazione accurata. 

I servizi offerti da piattaforme cloud sono IaaS, \textit{Infrastructure as a Service}, macchine virtuali, risorse di massa, firewall, tutti realizzati virtualmente; PaaS \textit{Platform as a Service}, la gestione di macchine virtuali database, risorse; Saas \textit{Software as a Service}, il software che viene eseguito su queste macchine virtuali può essere comprato o affittato direttamente da queste piattaforme cloud. 
Altri servizi specializzati possono offrire calcolo massivi. Alcuni servizi di cloud sono reti pubbliche, altri possono essere privati o ibridi. 
I leader del settore sono AWS, \textit{Amazon Web Services}, Microsoft Azure e Google Cloud. Questi cloud possono popolare diversi data center, AWS si trova su 25 regioni, in questo caso macro-regioni geografiche. Microsoft si trova su più di 60 regioni mentre Google principalmente in America del Nord. 


Quando si compra un servizio su una piattaforma cloud, questo è sparo non necessariamente su uno stesso data center, poiché per trasferire una macchina virtuale è estremamente facile. Le risorse acquistate su piattaforme cloud sono distribuite ed in base alla necessità del gestore possono essere trasferite facilmente. Questo vale sia per macchine virtuali che per la memoria di massa. 

In Europa è stato creato un sistema federato di ISP europei, GAIA-X, per realizzare hub nazionali indipendenti dall'hardware per garantire servizi specificando un modello con garanzia di controllo sulla locazione delle macchine e la strada che percorrono i dati. 



In Italia l'Agenzia per la Cybersicurezza Nazionale ha fondato un cloud marketplace per offrire alla pubblica amministrazione di acquistare risorse cloud da parte di provider qualificati. La CONSIP ha un servizio di supporto per i servizi cloud dedicati alla pubblica amministrazione. 
Inoltre c'è un programma per portare ed abilitare la pubblica amministrazione all'uso dei servizi cloud. 
Questo è previsto dal PNRR, nel primo obiettivo, attraverso una migrazione ad un cloud nazionale PSN, oppure ad un provider certificato. Anche la stessa migrazione ad un altro cloud viene trattato come un servizio senza dover necessariamente conoscere il suo funzionamento tecnico. 


\end{document}